\documentclass{article}
\usepackage[english]{babel}
\usepackage{minutes}
\usepackage[letterpaper, margin=1in]{geometry}

\pagestyle{plain}

\setlength{\parindent}{0em}
\setlength{\parskip}{1em}

\begin{document}
\selectlanguage{english}

\begin{Minutes}{Initial Supervisor Meeting}
\participant{\
    Emily Shepherd, \
    Mohit Gupta, \
    Dan Playle, \
    Toby Finch, \
    Calin Pasat, \
    Dr Geoff Merrett, \
    Dr Alex Weddell}
\minutesdate{30 September 2015}
\starttime{15:00}
\endtime{16:00}
\location{59/4227}
\maketitle

\topic{Introduction}
Dr Merrett led the meeting; he began by introducing himself and Dr
Weddell. He then went over the current
proposal.
He noted that it had
been primarily aimed at Electronic students, so there may need to be
changes in light of the fact three of the team are Computer
Scientists, but he did not think this would be a problem. He also stated
that the project was aimed at a team of 4 and there might, therefore, be
room for extensions.

After the project brief, Dr Merrett asked the team members what their
primary interests were. Emily began by stating that the team as a whole
had asked for a Networking-related task. Each member of the team
concurred, Security also being mentioned by several members. Dr Merrett
stated that those interests would not be compatible with the brief, but
there was a general consensus that this would not be a problem going
forward.

Following from this, Dr Merrett asked if the team had any questions
about the brief. Emily responded that his
description had been very good, and that there was scope in the project
to use both the Computer Scientists' and Electronic Engineers' skill sets.

\topic{Hardware}
Dr Weddell asked if the team were aware what an M0 was (this is referred
to in the Project Brief) - to which the general response was no. He then
explained that the Cortex M class processors are designed to work in
low-power systems and environments and that the M0 is the lowest in this
class. Dr Merrett gave a more general description of the concept of
reducing power at the cost of having to run the processor more slowly.

Dr Weddell stated that it would be sensible to use an under-clocked M0
for our project, to emulate the capability of the proposed ultra-low
power processor. Dr Merrett stated that the group should acquire an FPGA Board
and a lab desk. Mohit and Calin confirmed they were able to do this, and
checked that a risk assessment would need to be carried out.

\task*{Acquire FPGA Board}
\task*{Complete Risk Assessment}
\task*{Acquire Lab Desk}

\topic{General GDP Discussion}
Dr Weddell pointed out that it would also be appropriate for the group to
inspect the mark scheme, to know what is best to focus on, and
noted that team work and project management contributed to a large
proportion of the marks. Dr Merrett agreed with this, and explained
that, while it would be easy for him to tell if his students had worked well as a team
as he would be meeting us regularly, it would not be so obvious to our
second examiner. He advised us, therefore, to take minutes of our meetings.

Dr Merrett further advised the group to delegate tasks into
sub-groups, and to nominate a leader, minute taker and budget holder. Dr
Weddell added that it was best for one member of the team to speak to
the company (ARM). Dr Merrett reminded the group that, should they wish to use
subjects to collect data about body movement, ethical approval would be
required and that this may take some time.

\task*{Take Formal Minutes of Meetings}
\task*{Nominate Leader}
\task*{Nominate Minute Taker}
\task*{Nominate Budget Holder}
\task*{Nominate Company Representative}
\task*{Investigate Ethics Approval}

\topic{Next Meeting}
Dr Merrett stated that the team should discuss the brief amongst
themselves, agreeing if it would require any extensions or alterations. If the team
wanted to change anything, this would require negotiation.

After some brief discussion, it was agreed that the best
time for future supervised meetings is 1pm on Thursdays.

\task*{Prepare Brief}
\decision*{Meet at 1pm on Thursdays}

\end{Minutes}
\end{document}