\documentclass{article}
\usepackage[letterpaper, margin=1in]{geometry}
\usepackage{pgfgantt}
\usepackage{pdflscape}

%% DRAFT %%
	\usepackage[printwatermark]{xwatermark}
	\usepackage{xcolor}
	\usepackage{graphicx}
	\usepackage{tikz}

	\newsavebox\mybox
	\savebox\mybox{\tikz[color=red,opacity=0.2]\node{DRAFT};}
	\newwatermark*[
		allpages,
		angle=45,
		scale=10,
		xpos=-30,
		ypos=15
	]{\usebox\mybox}
%% END DRAFT %%

\title{School of Electronics And Computer Science \\
ELEC6200 MEng Group Design Project \\
Project Specification And Plan}
\author{Group 2}
\date{9 October 2015}

\setlength{\parindent}{0em}
\setlength{\parskip}{1em}

\begin{document}

\maketitle

\textbf{Title:} Ultra-low-power exercise monitoring applications for sub-threshold micro-controllers

\textbf{Supervisors:} Dr Geoff Merrett and Dr Alex Weddell

\textbf{Team Members:}
\begin{itemize}
	\item Emily Shepherd (ams2g11)
	\item Mohit Gupta (mg8g12)
	\item Toby Finch (tlf1g12)
	\item Dan Playle (djap1g11)
	\item Calin Pasat (cp10g12)
\end{itemize}

\textbf{Customer:} James Myers, ARM Research, Cambridge (James.Myers@arm.com)

\textbf{Project Specification:}

This project will research and attempt to develop a suitable algorithm for identifying and
monitoring excersies performed by a human wearer, for execution on an ultra-low-power
(a few uW) subthreshold ARM Cortex-M0+, which is currently being investigated at ARM Research.
This requires that the algorithm can execute on a processor clocked at a few hundred
kHz and from limited memory.

The project shall be split into three overlapping areas. The first task will be to effectively
emulate the required environment, by creating a test platform from an ARM DesignStart Cortex-M0
to an FPGA. This must be clocked at frequencies from 100s of kHz to a few MHz, and contain a
single memory space of 32kB for use by both program code and variables. 

Secondly, the project shall investigate existing algorithms and approaches for monitoring
exercises and activities, and develop a algorithm optimsed for the excerises performed to avoid
suffering from deep vein thrombosis
\footnote{For example:
http://www.virgin-atlantic.com/gb/en/travel-information/your-health/inflight-exercise.html}
and suitable for the target platform.

Finally, the theorised algorithm will be implemented and deployed onto the test platform, allowing
for demonstration and evaluation, including making estimates of the energy and power consumption
compared to existing Service oriented Architecture approaches.

\newpage
\begin{landscape}

\textbf{Project Plan:}

\textit{It is unclear which seminar sessions we will be in. For the purposes of planning, we have
assumed we will have the earlier sessions.}

\begin{ganttchart}[
		hgrid,
		vgrid,
		x unit=4mm,
		y unit chart=.7cm,
		time slot format=isodate
	]{2015-09-28}{2015-11-15}
	\gantttitlecalendar{month=shortname, week, day} \\

	\ganttbar{Agree Project Brief}{2015-09-30}{2015-10-08} \\
	\ganttlinkedmilestone{Project Brief Due}{2015-10-09} \\

	\ganttbar{Seminar Preparation}{2015-10-19}{2015-10-22} \\
	\ganttlinkedmilestone{Progress Seminar 1}{2015-10-23} \ganttnewline[ultra thick, red]

	\ganttgroup{Hardware Work}{2015-10-02}{2015-11-01} \\
	\ganttbar{Aquire Lab Desk}{2015-10-02}{2015-10-12} \\
	\ganttbar{Aquire FPGA Board}{2015-10-05}{2015-10-10} \\
	\ganttbar{Do some work and shit}{2015-10-13}{2015-11-01} \ganttnewline[ultra thick, red]

	\ganttgroup{Movement Data Study}{2015-10-05}{2015-10-25} \\
	\ganttbar{Search for Volunteers}{2015-10-12}{2015-10-21} \\
	\ganttbar{Apply for Ethics Approval}{2015-10-05}{2015-10-18} \\
	\ganttlinkedbar{Collect Movement Data}{2015-10-19}{2015-10-25} \ganttnewline[ultra thick, red]

	\ganttgroup{Algorithm Work}{2015-10-05}{2015-11-15} \\
	\ganttbar{Background Research}{2015-10-05}{2015-10-25} \\
	\ganttbar{Algorithm Design}{2015-10-19}{2015-11-15} \\
	\ganttbar{Algoritm Implementation}{2015-11-02}{2015-11-15}

	
\end{ganttchart}

\begin{ganttchart}[
		hgrid,
		vgrid,
		x unit=4mm,
		y unit chart=.7cm,
		time slot format=isodate
	]{2015-11-16}{2016-01-03}
	\gantttitlecalendar{month=shortname, week=8, day} \\

	\ganttbar[
		inline,
		bar height=1,
		bar top shift=0,
		bar/.append style={fill=red, draw=red},
		bar label font=\color{yellow}
	]{Christmas Holidays}{2015-12-12}{2016-01-03} \\

	\ganttbar{Seminar Preparation}{2015-11-16}{2015-11-19} \\
	\ganttlinkedmilestone{Progress Seminar 2}{2015-11-20} \\

	\ganttbar{Report Writing}{2015-12-14}{2015-12-20}
	\ganttbar{}{2015-12-28}{2015-12-30} \ganttnewline[ultra thick, red]
	
	\ganttbar{Algorithm Implementation}{2015-11-16}{2015-11-29} \\
	\ganttbar{Testing and Evaluation}{2015-11-16}{2015-12-09} \\
	\ganttmilestone{Target Completion}{2015-12-09}
\end{ganttchart}

\newpage

\textit{The Reflection Report's handin date is included for informational purposes only.
As each team member must complete their own report, the time to complete this has
not been timetabled in the Gantt chart. Instead, each member will plan this individually.}

\begin{ganttchart}[
		hgrid,
		vgrid,
		x unit=4mm,
		y unit chart=.7cm,
		time slot format=isodate
	]{2016-01-04}{2016-02-21}
	\gantttitlecalendar{month=shortname, week=15, day} \\

	\ganttbar[
		inline,
		bar height=1,
		bar top shift=0,
		bar/.append style={fill=blue, draw=blue},
		bar label font=\color{yellow}
	]{Exams}{2016-01-11}{2016-01-22} \\

	\ganttbar{}{2016-01-04}{2016-01-10}
	\ganttbar{Report Writing}{2016-01-23}{2016-01-27} \\
	\ganttlinkedmilestone{Report Due}{2016-01-28} \\

	\ganttmilestone{Reflection Report Due}{2016-02-01} \\

	\ganttbar{Poster and slides work}{2016-01-23}{2016-02-03} \\
	\ganttlinkedmilestone{Poster and Slides Due}{2016-02-04} \\

	\ganttbar{Presentation Preparation}{2016-02-08}{2016-02-19} \\
	\ganttlinkedmilestone{Final Presentation}{2016-02-20}
\end{ganttchart}

\end{landscape}

\end{document}
