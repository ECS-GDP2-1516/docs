\documentclass{article}
\usepackage[letterpaper, margin=0.7in]{geometry}
\usepackage{pgfgantt}
\usepackage{pdflscape}

\title{School of Electronics And Computer Science \\
ELEC6200 MEng Group Design Project \\
Project Specification And Plan}
\author{Group 2}
\date{9 October 2015}

\setlength{\parindent}{0em}
\setlength{\parskip}{1em}

\begin{document}

\maketitle

\textbf{Title:} Ultra-low-power exercise monitoring applications for sub-threshold micro-controllers

\textbf{Supervisors:} Dr Geoff Merrett and Dr Alex Weddell

\textbf{Team Members:}
\begin{itemize}
	\item Emily Shepherd (ams2g11)
	\item Mohit Gupta (mg8g12)
	\item Toby Finch (tlf1g12)
	\item Dan Playle (djap1g11)
	\item Calin Pasat (cp10g12)
\end{itemize}

\textbf{Customer:} James Myers, ARM Research, Cambridge (James.Myers@arm.com)

\textbf{Project Specification:}

This project will research and attempt to develop an algorithm for identifying and
monitoring exercises performed by a human wearer, suitable for execution on an ultra-low-power
(a few uW) subthreshold ARM Cortex-M0+, which is currently being investigated at ARM Research.
This requires that the algorithm can execute on a processor clocked at a few hundred
kHz and from limited memory.

The project shall be split into three overlapping areas. The first task will be to effectively
emulate the required environment, by creating a test platform from an ARM DesignStart Cortex-M0
to an FPGA. This must be clocked at frequencies from 100s of kHz to a few MHz, and contain a
single memory space of 32kB for use by both program code and variables. This will then be used to
obtain movement data for test subject performing exercises designed to reduce the risk of suffering from
deep vein thrombosis whilst flying\footnote{For example:
http://www.virgin-atlantic.com/gb/en/travel-information/your-health/inflight-exercise.html}.

Secondly, the project shall investigate existing algorithms and approaches for monitoring
exercises and activities, specifically focusing on following machine learning
approaches. Existing tools, such as Weka, will be considered to assist with assessing of machine learning approaches. At first the research will look at unconstrained systems, before looking at
machine learning in a wider context on constrained systems and how effective algorithms can
deal with a power-effectiveness tradeoff. Finally the group will aim to use this research
to inform the conception of a system designed for deep vein thrombosis exercises.

Finally, the theorised algorithm will be implemented and deployed onto the test platform, allowing
for demonstration and evaluation, including making estimates of the energy and power consumption
compared to the existing algorithms looked at in the research phase.

\begin{landscape}

\begin{ganttchart}[
		hgrid,
		vgrid,
		x unit=4mm,
		y unit chart=.7cm,
		time slot format=isodate
	]{2015-09-28}{2015-11-15}
	\gantttitlecalendar{month=shortname, week, day} \\

	\ganttbar{Agree Project Brief}{2015-09-30}{2015-10-08} \\
	\ganttlinkedmilestone{Project Brief Due}{2015-10-09} \\

	\ganttbar{Seminar Preparation}{2015-10-19}{2015-10-25} \\
	\ganttlinkedmilestone{Progress Seminar 1}{2015-10-26} \ganttnewline[ultra thick, red]

	\ganttgroup{Hardware Work}{2015-10-02}{2015-11-01} \\
	\ganttbar{Acquire Lab Desk}{2015-10-02}{2015-10-09} \\
	\ganttbar{Acquire FPGA Board}{2015-10-05}{2015-10-10} \\
	\ganttbar{Obtain Cortex M0 verilog}{2015-10-05}{2015-10-10} \\
	\ganttlinkedbar{Cortex M0 on FPGA}{2015-10-11}{2015-10-17} \\
	\ganttbar{Design hardware schematic}{2015-10-05}{2015-10-15} \\
	\ganttbar{Buy required components}{2015-10-12}{2015-10-24} \\
	\ganttlinkedbar{Build hardware}{2015-10-15}{2015-11-01} \\
	\ganttbar{Sensor Code}{2015-10-19}{2015-10-30} \ganttnewline[ultra thick, red]

	\ganttgroup{Movement Data Study}{2015-10-05}{2015-10-30} \\
	\ganttbar{Search for Volunteers}{2015-10-12}{2015-10-23} \\
	\ganttbar{Apply for Ethics Approval}{2015-10-05}{2015-10-23} \\
	\ganttlinkedbar{Collect Movement Data}{2015-10-26}{2015-10-30} \ganttnewline[ultra thick, red]

	\ganttgroup{Algorithm Work}{2015-10-05}{2015-11-15} \\
	\ganttbar{Background Research}{2015-10-05}{2015-10-25} \\
	\ganttbar{Algorithm Design}{2015-10-19}{2015-11-15} \\
	\ganttbar{Algorithm Implementation}{2015-11-02}{2015-11-15}

	
\end{ganttchart}

\begin{flushright}

\begin{ganttchart}[
		hgrid,
		vgrid,
		x unit=4mm,
		y unit chart=.7cm,
		time slot format=isodate
	]{2015-11-16}{2016-01-03}
	\gantttitlecalendar{month=shortname, week=8, day} \\

	\ganttbar[
		inline,
		bar height=1,
		bar top shift=0,
		bar/.append style={fill=red, draw=red},
		bar label font=\color{yellow}
	]{Christmas Holidays}{2015-12-12}{2016-01-03} \\

	\ganttbar{Seminar Preparation}{2015-11-16}{2015-11-22} \\
	\ganttlinkedmilestone{Progress Seminar 2}{2015-11-23} \\

	\ganttbar{Report Writing}{2015-12-14}{2015-12-20}
	\ganttbar{}{2015-12-28}{2015-12-30} \ganttnewline[ultra thick, red]
	
	\ganttbar{Algorithm Implementation}{2015-11-16}{2015-11-29} \\
	\ganttbar{Testing and Evaluation}{2015-11-16}{2015-12-09} \\
	\ganttmilestone{Target Completion}{2015-12-09}
\end{ganttchart}

\begin{ganttchart}[
		hgrid,
		vgrid,
		x unit=4mm,
		y unit chart=.7cm,
		time slot format=isodate
	]{2016-01-04}{2016-02-21}
	\gantttitlecalendar{month=shortname, week=15, day} \\

	\ganttbar[
		inline,
		bar height=1,
		bar top shift=0,
		bar/.append style={fill=blue, draw=blue},
		bar label font=\color{yellow}
	]{Exams}{2016-01-11}{2016-01-22} \\

	\ganttbar{}{2016-01-04}{2016-01-10}
	\ganttbar{Report Writing}{2016-01-23}{2016-01-27} \\
	\ganttlinkedmilestone{Report Due}{2016-01-28} \\

	\ganttmilestone{Reflection Report Due}{2016-02-01} \\

	\ganttbar{Poster and slides work}{2016-01-23}{2016-02-03} \\
	\ganttlinkedmilestone{Poster and Slides Due}{2016-02-04} \\

	\ganttbar{Presentation Preparation}{2016-02-08}{2016-02-19} \\
	\ganttlinkedmilestone{Final Presentation}{2016-02-20}
\end{ganttchart}

\end{flushright}

\end{landscape}

\end{document}
