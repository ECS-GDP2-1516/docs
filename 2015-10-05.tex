\documentclass{article}
\usepackage[english]{babel}
\usepackage{minutes}
\usepackage[letterpaper, margin=1in]{geometry}

\pagestyle{plain}

\setlength{\parindent}{0em}
\setlength{\parskip}{1em}

\begin{document}
\selectlanguage{english}

\begin{Minutes}{Team Meeting 2}
\participant{
    Emily Shepherd,
    Mohit Gupta,
    Dan Playle,
    Toby Finch,
    Calin Pasat}
\minutesdate{5 October 2015}
\starttime{15:00}
\endtime{15:40}
\location{Zepler Level 3 Labs}
\maketitle

\topic{Project Brief}

Emily presented the draft project brief and asked if anyone had any notes about the text -
the group had none and it was unanimously approved.

Emily then moved onto the Gantt chart section - she opened by suggesting that the ``Hardware Work''
section seemed to be doing too much in the first few weeks to be realising. Dan agreed, and then pointed
out that the movement study should also be pushed back, as it would not be possible to do this until
after they had the sensors working.

Emily then pointed out that a lot of the hardware work seemed to be
taking place before the gantt chart showed the hardware being acquired. Mohit explained that he would be
able to start the work before some of the parts arrived, and the time shown on the Gantt chart was
worst case scenario if the vendors are slow. Dan said he was aware of a vendor who he believed could
provide next day delivery, however Mohit pointed out that they would need to check that it was on the
approved list. Emily suggested leaving the time to acquire hardware at a worst case scenario of two weeks
and asked how long after this it would be ready for the movement study - Dan and Mohit both agreed that
it could be ready a day after, albeit in a very rough state. Emily suggested leaving a weekend on the
plan, which the team were happy with.

Dan noted that he wasn't sure about the ``Develop Embedded Code'' section, and Emily suggested it simply
be combined with ``Algorithm Implementation'' - Mohit explained he had kept the sections distinct to
separate the code which deals with the sensors and the code which runs the algorithm, however Emily said
she felt that the ``software'' could put together. Dan pointed out that it was very difficult to give
an accurate timescale for many of the items until there was some research and data to guide the process, and
Mohit and Toby agreed that the plan could change as the project went on.

\decision*{Push back movement study}

\topic{Ethical Approval}

Both Emily and Toby had looked into this - Emily began by stating that she had created a submission on
the ERGO website and filled out the first required form. Toby said that he believed that the study
would be classified as ``intrusive'' which would require the team to use consent forms, which in turn
would require a Data Protection Plan, as a signature counts as personal information. He then stated that
the team would need to carefully plan the research as this would need to be submitted with the application.

Dan asked if the participants would be used to verify that the device worked at the end of the project as
well as collecting the movement data at the start - Emily said she had also considered this, but wasn't
sure if there would be time. Calin and Toby stated that they thought it would be worth doing, and Toby
suggested that it might be worth applying for if there is time - Mohit suggested adding it as a possible
extension.

Emily then suggested the team discuss exactly how the research would be carried out - Dan and Mohit pointed
out that this would be difficult this early in the process, as they did not yet know how many sensors they
would be able to acquire. After looking at the required exercises and confirming that the device would need
to measure more than simply feet movement, Dan suggested simply asking the users to move the sensor to the
active body part between exercises but Emily and Mohit both agreed that that would be both inconvenient. It
was agreed to assess the budget and approve the project to decide how many sensors would be required.

Emily then reminded the group that, as the research is to be ``intrusive'', a risk assessment would be
required and asked the group to consider what possible risks there may be. The team identified the following
issues:

\begin{itemize}
	\item Electrocution from wiring failure in the device or power line
	\item Heat or fire from electrical issue
	\item Constriction and blood flow disruption if device is too tight
	\item As the initial version may be quite rough, it may not be self powered, causing a trip and choking hazard from dangling cables
	\item Strains from the exercise
		\begin{itemize}
			\item Mohit suggested keeping a first aid kit on hand
		\end{itemize}
\end{itemize}

Toby then pointed out that debrief form is also required which should include contact info, a way to thank
participants and information for the candidates to access the results when complete. Calin asked how it
would be possible to send the results to the participants if they are to be anonymous - Emily suggested
simply giving a link which will contain the data once the results are available, they can then check
themselves. Calin asked if it would be possible to simply ask if the participants if they would like to be
contacted.

Toby asked if ARM were funding the project, Emily confirmed that they aren't. Toby then pointed out that
the ethics approval will take the group two weeks - Emily said she should update the Gantt chart to three
weeks to reflect this, one for the team to complete the request, and two for the ethics committee to review
it.

\decision*{Extend ``Apply for Ethics Approval'' bar on the Gantt chart}
\task{Emily, Toby}{Complete Ethics Approval Request}

\topic{Lab Desks}

Mohit informed the group that each member of the team would need to fill out a risk assessment form to gain
access to the lab. The group agreed to complete these at the end of the meeting and give these to Emily, who
would bring them all to their next meeting with Dr Merrett to be countersigned.

\task{Emily}{Bring forms to next meeting}

\topic{Git}

Emily told the group she had created a git repository under her own GitHub account for minutes, but stated
that the team would clearly need at least one more for code related items. She asked if the team were happy
to continue using her account.

Dan asked the team which branching strategy they preferred, and immediately expressed a preference for the
use of feature branches. Emily and Mohit agreed strongly, Mohit stating that we should absolutely refrain
from committing directly to the master branch. Dan added that, as such, we should not use fast forwards,
which Emily agreed with. Emily suggested that this formal branching strategy would not be required for the
minutes repository.

Emily stated that she would be signing each of her commits.

Calin noted that he had not used git before, but had signed up for an account on GitHub and had begun
checking out how it works. Emily said she would be happy to give him a crash course, and Dan added that
he was welcome to ask the whole team questions, which Mohit agreed with.

\decision*{Continue using current repository setup}
\decision*{Use feature branches for code repository, with no fast fowards}
\task{Calin}{Become familiar with Git}

\topic{Background Research}

Emily pointed out that the gantt chart had background research starting this week and suggested starting
by looking at Machine Learning in the context of exercise recognition, which Dan agreed with. Emily asked
if the project would only use offline learning or if it would be feasible to employ online learning on the
device - Dan was not sure if this would be possible, however Mohit, Toby and Calin though that live learning
would be an important aspect. Calin and Mohit both suggested that at least some form of calibration phase
should be looked at. Emily theorised that the system could have a calibration phase once, which may require
more power, but then could then go into a lower power ``usage phase'' for day normal usage.

This prompted a discussion on the machine's memory - the specification had said 32kB would be availiable,
however it had been assumed by the group that this would be volatile memory. It was unclear how much, if
any, long term memory would be available. Emily pointed out that, as the a proposed usecase was on planes,
it would be sensible to have longer term memory in place, as it would not be practical to download the data
during a flight.

\end{Minutes}
\end{document}